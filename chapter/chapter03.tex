\chapter{基于空间邻接感知的模块化建筑平面布局优化研究}
\label{chap:layout_optimization}

\section{引言}
\label{sec:layout_intro}

\subsection{研究背景:模块化建筑平面设计的范式转变}

模块化建筑(Modular Construction)作为建筑工业化的高级形态,凭借其在提
升施工效率、优化质量控制以及降低环境足迹方面的显著优势,近年来在全球范围内得
到了广泛关注与应用 \citep{Lawson2014, zhai2019internet}。与传统的现场浇
筑模式不同,模块化建筑将大部分建造工序前置于受控的工厂环境中完成,这种制造模
式的革新对上游的设计流程提出了全新的约束与挑战 \citep{thai2020review}。

在传统建筑设计中,设计师通常遵循“形式追随功能”的原则,优先考虑空间的功
能组织、流线效率及形态美学。然而,在模块化建筑的语境下,设计逻辑发生了根本性
的转变,要求“功能适应模数”。这意味着建筑平面不仅要满足使用功能,还必须能
够被精确地离散化(Discretized)为一系列符合运输、吊装及制造限制的标准化体
积单元(Volumetric Units)。

这种从连续的自由形态平面到离散的模块化组合的转换过程,即“平面
模块化”(Floor Plan Modularization),本质上是一个复杂的几何分
割与组合优化问题。如图 \ref{fig:conceptual_framework} 所示,该过程旨
在将任意给定的建筑平面轮廓自动转化为由标准化模块组成的网格布局。一个高质量
的模块化布局方案不仅需要最大化地覆盖原始轮廓以保障得房率,更需要严格遵守模
块的尺寸约束,并维持房间内部的空间完整性(Spatial Integrity)与房间之间
的功能邻接关系(Spatial Adjacency)。


\begin{figure}[H]
\centering
\includegraphics[width=1\textwidth,height=0.4\textheight, keepaspectratio]{Fig2/test1.png}
\caption{本文提出的空间邻接感知模块化过程的概念框架示意图。}
\label{fig:conceptual_framework}
\end{figure}

\subsection{现有方法的局限性}

尽管计算机辅助设计(CAD)和建筑信息模型(BIM)技术已广泛普及,但目前
的模块化平面设计仍主要依赖人工经验或半自动化的规则算法,效率低下且难以保
证解的质量。现有的自动化布局方法主要存在以下局限性:

\begin{enumerate}
    \item \textbf{缺乏对空间拓扑关系的考量:} 传统的几何分割算
	法(如矩形剖分、网格平铺)通常将平面视为纯粹的几何图形进
	行切割 \citep{Wu2019}。这种“几何优先”的策略往往忽视了建筑功
	能的内在逻辑,容易切断房间之间的必要连接(例如将一个完整的起居室
	分割为互不连通的两个模块),或者产生大量无法使用的狭长碎片空间,导致
	生成的布局在建筑功能上不可行。
    
    \item \textbf{标准化程度与灵活性的矛盾:} 现有的启发式优化
	算法(如遗传算法、模拟退火)在处理规则矩形轮廓时表现尚可,但
	在面对现实中常见的非凸、不规则建筑边界时,往往难以在“模块标准
	化”(减少模块类型以降低成本)和“轮廓匹配度”(提高覆盖率)之间
	找到最佳平衡点,容易陷入局部最优解。
    
    \item \textbf{计算效率瓶颈:} 许多基于搜索的生成方法随着平面复
	杂度的增加,计算时间呈指数级增长,难以满足早期设计阶段快速迭代、
	实时反馈的需求。
\end{enumerate}

\subsection{本章研究目标与主要贡献}

针对上述挑战,本章提出了一种基于深度强化学习(Deep Reinforc
ement Learning, DRL)的端到端模块化布局生成框架。该框架不依赖于
预定义的分割模板,而是通过学习“如何组装”的策略,从底层的原子网格出
发,逐步构建出最优的模块化布局。

本章的主要研究贡献总结如下:

\begin{itemize}
    \item \textbf{提出基于空间邻接感知的状
	态表示(Adjacency-Aware State Representation):} 创新性地构建了
	包含几何特征与拓扑关系的多模态状态空间。通过引入邻接
	图(Adjacency Graph)作为状态的一部分,智能体在决
	策过程中能够显式地感知当前网格与周边区域的连通性,从而有
	效防止功能空间的割裂。

    \item \textbf{设计基于“合并”策略的动作空
	间(Merge-based Action Space):} 不同于传统的“自顶向下”
	分割策略,本研究采用“自底向上”的合并机制。智能体从细粒度的正交
	网格开始,通过学习最优的合并动作序列,将相邻网格聚合成符合尺寸
	约束的标准化大模块。这种机制天然地保证了模块的几何规整性。
    
    \item \textbf{构建多目标奖励函数体
	系(Multi-objective Reward Function):} 综合考虑
	了覆盖率(Coverage)、模块数量(Module Count)、类型
	标准化(Standardization)以及边界平滑度(Boundary Smoothness)等多个
	优化目标。通过精心设计的奖励信号,引导智能体在复杂的约束空间中
	搜索出兼顾制造效率与建筑性能的全局最优解。
\end{itemize}

通过本章提出的方法,我们致力于解决“标准化制造”与“多样化设计”之间
的矛盾,为模块化建筑的智能设计提供一种新的技术路径。

\section{模块化布局生成的问题定义与建模}
\label{sec:problem_formulation}

在深入探讨具体的算法实现之前,本节首先对模块化平面布局生成问题进行严
格的数学定义。我们将该问题建模为一个受多重几何与拓扑约束限制的组合优化
问题(Combinatorial Optimization Problem),旨在寻找一组最优的模块
配置,使其在满足制造与功能要求的同时,最大化地匹配给定的建筑轮廓。

\subsection{符号定义与输入输出}

给定一个任意形状的建筑平面轮廓 $\mathcal{P}$,该轮廓通常由一组有
序的顶点 $V = \{v_1, v_2, \dots, v_k\}$ 定义的多边形边界
表示。我们的目标是生成一个模块集合 $\mathcal{M} = \{m_1, m_2, \dots, m_n\}$,其中
每一个 $m_i$ 代表一个矩形的模块化单元。

每个模块 $m_i$ 可以由其几何属性唯一确定,记为五元组:
\begin{equation}
    m_i = (x_i, y_i, w_i, h_i, \theta_i)
\end{equation}
其中,$(x_i, y_i)$ 表示模块的中心坐标,$w_i$ 和 $h_i$ 分别
表示模块的宽度和长度,$\theta_i \in \{0, 90^\circ\}$ 表示
模块的旋转角度(通常限制为正交旋转以适应标准化制造)。

\subsection{约束条件建模}

为了保证生成的布局方案在工程上的可行性,模块集合 $\mathcal{M}$ 必须
满足以下三类核心约束:

\subsubsection{几何覆盖约束 (Geometric Constraint)}
生成的模块必须严格位于给定的建筑边界内部,且不能超出轮廓范围。这一约
束确保了设计方案不违反用地的红线限制。形式化表示为:
\begin{equation}
    \forall m_i \in \mathcal{M}, \quad \text{Area}(m_i \cap \mathcal{P}) = \text{Area}(m_i)
    \label{eq:boundary_constraint}
\end{equation}
即模块 $m_i$ 与轮廓 $\mathcal{P}$ 的交集面积必须等于模块自身的面积。

\subsubsection{非重叠约束 (Non-overlapping Constraint)}
在物理空间中,任意两个模块之间不能发生体积上的冲突。即对于任意
两个不同的模块 $m_i$ 和 $m_j$,它们的空间交集必须为空:
\begin{equation}
    \forall i \neq j, \quad m_i \cap m_j = \emptyset
    \label{eq:overlap_constraint}
\end{equation}
这一约束保证了布局的物理合法性,避免了结构冲突。

\subsubsection{模数与尺寸约束 (Dimensional Constraint)}
模块化建筑的核心优势在于标准化。因此,模块的尺寸不能是任意
连续值,而必须属于一个预定义的标准化尺寸集合 $\mathcal{S}$。该集合
通常由运输限制(如道路限宽、集装箱尺寸)和生产线模数决定:
\begin{equation}
    (w_i, h_i) \in \mathcal{S} \quad \text{or} \quad (h_i, w_i) \in \mathcal{S}
    \label{eq:dimension_constraint}
\end{equation}
例如,$\mathcal{S}$ 可能包含 $\{3m \times 6m, 3m \times 9m, \dots\}$ 等标准规格
。违反此约束将导致定制成本激增,丧失模块化的经济性。

模块化建筑布局优化面临着若干相互关联的挑战,需要细致的解决方案。首要挑战在于捕捉和利
用模块之间复杂的空间邻接关系——传统优化方法往往过度简化这些关系,阻碍了真正高效布局
的生成。此外,优化过程需要平衡模块整合、空间效率、几何规则性和制造约束等内在冲突的目
标,这对多目标处理能力提出了极高要求。

由于模块化布局设计固有的解空间呈指数级增长,使得穷举搜索变得不可行,因此必须采用高效的
探索策略。这种复杂性与经典的理论发现相一致,即“将带孔的直线多边形划分为最少数量的矩
形”属于 **NP-难(NP-hard)问题** \citep{1979Computers}。最后,优化后的布局必须严格遵
守实际的制造和运输限制(如最大模数尺寸和完美矩形要求),这进一步增加了优化的难度。

现有的模块化布局优化方法主要依赖于传统优化算法,在应对上述复杂挑战时表现出明显的
局限性。遗传算法(GAs)和其他进化方法虽然具备全局搜索能力,但往往难以有效地捕捉
和利用建筑模块之间细微的空间邻接关系。这些传统方法通常将优化问题视为静态的数学公
式,缺乏处理空间设计问题动态性和情境依赖性所需的自适应决策能力。

为了解决这些局限性,本研究采用了一种纯粹的强化学习方法,利用深度 Q 网络(DQN)来实
现高效的模块化建筑布局优化。该方法通过以下四个关键创新得以实现:
\begin{enumerate}
    \item 建立了专门的空间邻接感知机制,确保合并决策在几何上可行且结构上合理;
    \item 设计了复杂的状态表示,全面捕捉布局特征,包括空间关系和形状复杂度指标;
    \item 构建了多目标奖励机制,在平衡相互冲突的目标的同时确保满足实际约束;
    \item 实施了严格的可行性验证流程,保证所有生成的布局均满足制造要求。
\end{enumerate}

\subsection{优化目标函数}

在满足上述硬性约束的前提下,我们旨在寻找一个“最优”的布局方案。这种最优性由以
下三个相互竞争的目标共同定义:

\begin{enumerate}
    \item \textbf{最大化覆盖率 (Coverage Maximization):} 为了提高土地利用率
	和得房率,模块集合 $\mathcal{M}$ 的总面积应尽可能接近建筑
	轮廓 $\mathcal{P}$ 的面积。
    \begin{equation}
        \max J_{\text{cov}} = \frac{\sum_{i=1}^{n} \text{Area}(m_i)}{\text{Area}(\mathcal{P})}
    \end{equation}
    
    \item \textbf{最小化模块数量 (Module Count Minimization):} 在覆盖率相
	同的情况下,使用更少的大尺寸模块通常优于使用大量小尺寸模块,因为这能减少现
	场吊装次数和拼缝处理工作量,从而降低施工成本。
    \begin{equation}
        \min J_{\text{num}} = n = |\mathcal{M}|
    \end{equation}
    
    \item \textbf{最大化标准化程度 (Standardization Maximization):} 为了发挥
	规模效应,应尽可能减少模块的类型数量(Types),即复用同一种规格的模块。
    \begin{equation}
        \min J_{\text{type}} = |\text{Unique}(\{(w_i, h_i) | m_i \in \mathcal{M}\})|
    \end{equation}
\end{enumerate}

综上所述,模块化平面布局生成问题被定义为一个多目标优化问题。由于建筑
轮廓的复杂性(非凸、不规则边界)以及离散的组合特性,该问题属于典
型的NP-难(NP-hard)问题,传统的精确求解算法难以在多项式时间内找
到全局最优解。这正是本章引入深度强化学习方法的根本动因。

\section{基于正交网格分解的初始化策略}
\label{sec:initialization}

为了将连续的建筑平面转化为计算机可处理的离散形式,并为后续的强化学习算法提供合法的状态空
间,本研究提出了一种基于正交网格分解(Orthogonal Grid Decomposition, OGD)的初始化策略
。为了将不规则的平面图转化为可管理的单元,我们采用了 Lin 等人 \citep{lin2025modulepacking} 提出
的两阶段划分策略(包含铡刀式划分和堆叠划分)。在本框架中,我们将这两个步骤
分别定义为正交网格分解(OGD)和模块尺寸归一化(MSN)。该策略旨在将任意形状
的建筑轮廓离散化为一系列基础的“原子单元”,作为模块化组装的最小颗粒度。

\subsection{轮廓离散化处理}

建筑平面轮廓 $\mathcal{P}$ 通常由一系列连续的二维坐标点定义。直接在连续空间中进行模块排布不仅计算复杂
度极高,而且难以保证模块间的无缝拼接。因此,首先需要对设计空间进行离散化处理。

初始阶段实施正交网格分解(OGD)方法,系统地将输入的正交多边形分解为一组完整的、
互不重叠的矩形子区域,为后续的模块化奠定基础。该过程包含以下步骤:
\begin{itemize}
    \item \textbf{边界特征提取:} 从多边形边界中提取所有不同的垂直和水平边缘坐标,形成潜在的网格结构。
    \item \textbf{网格线生成:} 基于提取的坐标生成全跨度的网格线——包括从上到下的垂直线和从左到右的水平线,覆盖整个多边形。
    \item \textbf{区域分割:} 首先应用所有垂直网格线创建垂直切片,随后利用水平网格线将这些切片进一步细分为矩形区域。
    \item \textbf{几何验证:} 确保每个生成的子区域都是具有正面积的有效矩形,消除分解过程中可能产生的任何退化形状或数值伪影。
\end{itemize}
这种分解方法保证了非重叠矩形对设计空间的完全覆盖,为后续的模块化优化过程提供了一组灵活的基元。

我们将建筑平面的包围盒(Bounding Box)划分为尺寸为 $\delta \times \delta$ 的正交网格系统。网格尺寸 $\delta$ 的设定至关重要,它通常取所有标准化模块尺寸的最大公约数(Greatest Common Divisor, GCD),以确保任何标准模块都可以由整数个网格单元组合而成。例如,若标准模块尺寸集合为 $\{3m \times 6m, 3m \times 9m\}$,则 $\delta$ 可设为 $3m$。

对于网格系统中的每一个单元 $g_{ij}$,我们根据其中心点 $c_{ij}$ 与建筑轮廓 $\mathcal{P}$ 的位置关系定义其状态 $S_{ij}$:
\begin{equation}
    S_{ij} = 
    \begin{cases} 
    1, & \text{if } c_{ij} \in \text{Interior}(\mathcal{P}) \\
    0, & \text{otherwise}
    \end{cases}
\end{equation}
其中,$S_{ij}=1$ 表示该网格为有效占据网格(Occupied Grid),$S_{ij}=0$ 表示无效网格。通过这一步骤,不规则的建筑轮廓被映射为一个二值化的占据矩阵(Occupancy Matrix)。

\subsection{初始原子网格生成}

在完成离散化后,所有标记为 $1$ 的有效网格构成了初始的“原
子网格集合”(Atomic Grid Set),记为 $\mathcal{A} = 
\{a_1, a_2, \dots, a_k\}$。每一个原子网格 $a_k$ 都被视为一
个独立的、最小尺寸的模块单元。

如图 \ref{fig:grid_decomposition} 所示,OGD 过程有效地解决
了非凸多边形(Non-convex Polygons)的填充难题。对于边界处的复
杂几何特征,算法采取“保守近似”策略,即仅保留中心点位于轮廓内
部的网格,从而保证生成的模块严格位于建筑红线范围内。

这一初始化过程为后续的深度强化学习任务构建了基础环境。此时,整个
平面布局被视为一个由 $k$ 个原子网格组成的初始状态,智能体的任务
即是通过一系列的决策动作,将这些细碎的原子网格逐步合并
为符合制造约束的大尺寸标准化模块。

% 请确认您 tex 文件中 Figure 2 的实际文件名,此处暂用 figs/grid_decomposition.pdf
\begin{figure}[htbp]
    \centering
    \includegraphics[width=1\textwidth]{Fig2/figure2.png}
    \caption{正交网格分解(OGD)过程示意图:(a) 原始建筑平面轮廓;(b) 网格离散化处理;(c) 生成的初始原子网格集合。该过程将连续的几何约束转化为离散的矩阵表示。}
    \label{fig:grid_decomposition}
\end{figure}

\subsection{模块尺寸归一化 (Modular Size Normalization)}
\label{subsec:msn}

正交网格分解(OGD)虽然能实现全覆盖,但可能产生超出制造限制的过大矩形。为了解
决这一问题,本研究在 OGD 之后引入了模块尺寸归一化(MSN)过程。

MSN 采用递归分割策略对尺寸超限的模块进行处理:
MSN 过程首先进行约束评估,将每个矩形与预定义的制造尺寸限制 $(W_{limit}, H_{limit})$ 进行比对。对于超出约束的矩形,执行双轴拆分评估:
\begin{itemize}
    \item 对于宽度超限的矩形,采用水平细分策略,确定均匀宽度的最佳段数,并生成相应的垂直分割线;
    \item 对于高度超限的矩形,通过计算可行的均匀高度段,创建水平分割线来解决。
\end{itemize}
拆分方向的选择基于最大化所得模块的粒度,以增强后续合并操作的灵活性。对于任何剩余的超大模块,将递归地应用此迭代细化过程。

经过 MSN 处理后,输出的模块集合 $\mathcal{M} = \{m_1, m_2, \dots, m_k\}$ 既保证
了对原始轮廓的几何覆盖,又严格符合预设的制造与运输尺寸限制,为后续的强化学习优化
提供了合法的初始状态空间。

\section{空间邻接感知的深度强化学习算法}
\label{sec:drl_algorithm}

基于上一节生成的初始原子网格,我们将模块化平面布局生成问题建
模为一个马尔可夫决策过程(Markov Decision Process, MDP)。本
节详细阐述该MDP的关键要素,包括状态空间、动作空间以及奖励函数的
设计,重点介绍如何通过引入拓扑信息来增强智能体对空间结构的感知能力。

\subsection{状态空间设计}
\label{subsec:state_space}

为了在保证计算效率的同时捕捉关键布局特征,不同于计算昂贵的图像输入,我们将智能体的状态 $s_t$ 封装为一个紧凑的**七维特征向量**。该向量编码了当前的几何统计信息与拓扑属性:

\begin{equation}
state = [N_m, N_t, A_a, A_{max}, A_{min}, A_{adj}, C_s]
\end{equation}

各分量定义如下:
\begin{itemize}
    \item \textbf{全局计数特征:} $N_m$ 为当前模块总数,反映合并进度;$N_t$ 为不同模块类型的数量,直接对应标准化程度(越低越好)。
    \item \textbf{面积统计特征:} 包含平均面积 $A_a$、最大面积 $A_{max}$ 和最小面积 $A_{min}$。这些指标帮助智能体识别过小的碎片区域或潜在的合并机会。
    \item \textbf{拓扑与形状特征:} $A_{adj}$ 为归一化的相邻对计数,衡量当前布局的合并潜力;$C_s$ 为平均形状复杂度,用于惩罚不规则形状。
\end{itemize}

这种低维向量表示法确保了智能体能够快速推理,避免了处理高维图像数据带来的计算开销。

\subsection{动作空间与合并机制}
\label{subsec:action_space}

与传统的“自顶向下”切割方法不同,本研究采用“自底向上”的合并
策略。动作空间 $\mathcal{A}$ 定义为所有合法的“模块合并”操
作的集合。

在时间步 $t$,智能体选择一个动作 $a_t$,即选择两个相邻的
模块 $m_i$ 和 $m_j$ 进行合并。为了保证生成的模块始终为矩形
(符合制造标准),合并操作必须满足**矩形约束(Rectang
ularity Constraint)**:即 $m_i$ 和 $m_j$ 合并后的形状必
须仍为一个矩形。

如图 \ref{fig:action_merge} 所示,若两个模块共享一条公
共边且合并后不形成凹多边形,则该合并动作是合法的。通过不断执
行合并动作,原本细碎的原子网格逐渐聚合成大尺寸的标准化模块,
直到没有合法的合并动作或达到终止条件为止。

% 请将此处的文件名 'figs/action_merge.png' 替换为您 tex 文件中 Figure 4 的实际文件名
\begin{figure}[htbp]
    \centering
    \includegraphics[width=0.8\textwidth]{Fig2/figure3.png}
    \caption{基于合并策略的动作空间示意图。智能体选择相邻的两个单元进行合并,前提是合并后的新单元必须保持矩形特征,以满足模块化制造的几何约束。}
    \label{fig:action_merge}
\end{figure}

\subsection{奖励函数设计}
\label{subsec:reward_function}

为了引导智能体在满足 DfMA(面向制造与装配的设计)原则的同时优化空间布局,我们设计了一个包含六个竞争目标的多目标奖励函数。该函数通过线性加权的方式平衡各优化目标:

\begin{equation}
\begin{split}
R = & \ w_{count} \cdot R_{count} + w_{type} \cdot R_{type} + w_{util} \cdot R_{util} \\
    & + w_{complexity} \cdot R_{complexity} + w_{compactness} \cdot R_{compactness} \\
    & + R_{validity} + R_{multi}
\end{split}
\label{eq:reward_function}
\end{equation}

其中各项的具体含义与权重配置如表 \ref{tab:hyperparameters} 所示。我们特别为 $R_{type}$ 分配了较高的权重(16.0),以显式地鼓励智能体优先考虑模块类型的标准化。此外,设置了严厉的无效惩罚 $R_{validity}$ (-50.0) 以确保几何可行性。

\begin{table}[htbp]
\centering
\caption{超参数设置与奖励函数权重配置}
\label{tab:hyperparameters}
\begin{tabular}{ll|ll}
\toprule
\multicolumn{2}{c|}{\textbf{训练超参数}} & \multicolumn{2}{c}{\textbf{奖励权重 (Eq. \ref{eq:reward_function})}} \\
\midrule
\textbf{参数} & \textbf{数值} & \textbf{符号} & \textbf{数值} \\
\midrule
批次大小 (Batch Size) & 32 & $w_{count}$ (数量减少) & 8.0 \\
经验池容量 (Replay Memory) & 10,000 & $w_{type}$ (类型标准化) & 16.0 \\
折扣因子 ($\gamma$) & 0.99 & $w_{util}$ (尺寸利用率) & 5.0 \\
学习率 ($\alpha$) & 0.001 & $w_{complexity}$ (形状简单度) & 2.0 \\
最大回合数 (Max Episodes) & 100 & $w_{compactness}$ (空间紧凑度) & 3.0 \\
无效惩罚 ($R_{validity}$) & -50.0 & 多重合并奖励 ($R_{multi}$) & 10.0 \\
\bottomrule
\end{tabular}
\end{table}


\subsection{网络架构与训练策略}
\label{subsec:network_training}

为了构建状态 $s_t$ 到动作 $a_t$ 的映射关系,本研究采用了深
度Q网络(Deep Q-Network, DQN)算法。DQN 结合了深度学习的
感知能力与强化学习的决策能力,能够有效处理高维状态空间下的
策略学习问题。

\subsection{神经网络架构}
\label{subsec:network_arch}

鉴于状态空间已被精简为特征向量,本研究采用了高效的前馈
神经网络(Feedforward Neural Network, FNN)作为 DQN 的函数近
似器,而非复杂的卷积网络。

如图 \ref{fig:dqn_architecture} 所示,网络架构具体配置如下:
\begin{itemize}
    \item \textbf{输入层:} 包含 7 个神经元,直接对应七维状态向量。
    \item \textbf{隐藏层:} 设计了两个全连接隐藏层。第一层包含 64 个神经元,第二层包含 32 个神经元。所有隐藏层均采用 ReLU \citep{nair2010rectified} 作为激活函数,以引入非线性映射能力。
    \item \textbf{输出层:} 输出节点的数量对应于当前状态下所有合法合并动作的数量。每个节点输出对应动作的 Q 值(即预期的长期累积奖励)。
\end{itemize}

该轻量级架构使得智能体能够在毫秒级内完成推理,满足了布局优化对实时性的要求。
为了在受限的几何环境中实现该架构,推理机制扩展了标准的 DQN 前向传播。
图 \ref{fig:decision_process} 展示了智能体的单步决策过程。与通用的 RL 实现不
同,本框架在动作选择阶段显式注入了空间邻接掩码(Spatial Adjacency Mask)。如图所
示,神经网络生成的原始 Q 值会经过邻接掩码的调制,过滤掉几何上无效的动作(例如合并非相邻模块)。这确保
了贪婪策略仅在有效动作子空间内运行。

\begin{figure}[H]
\centering
\includegraphics[width=1\textwidth, keepaspectratio]{Fig2/figure_new.png} 
\caption{DQN 智能体的单步决策过程示意图。}
\label{fig:decision_process}
\end{figure}

\subsubsection{训练策略}
为了提高训练的稳定性和收敛速度,本研究采用了以下策略:
\begin{itemize}
    \item \textbf{$\epsilon$-贪婪策略($\epsilon$-gr
    eedy Policy):} 在动作选择阶段,以概率 $\epsilon$ 随机
    选择动作以保持探索性(Exploration),以概率 $1-\epsilon$ 选
    择当前Q值最大的动作以利用已知经验(Exploitation)。随着训
    练的进行,$\epsilon$ 值逐渐减小,使智能体从探索逐渐转向利用。
    \item \textbf{经验回放(Experience Replay):} 构建经验
    池存储智能体的交互样本 $(s_t, a_t, r_t, s_{t+1})$。训练时
    从中随机采样小批量数据(Mini-batch)进行梯度下降更新。这
    一机制打破了样本间的时序相关性,显著提升了数据利用率和训练
    稳定性。
    \item \textbf{目标网络(Target Network):} 引入一个独立
    的、参数更新滞后的目标网络 $\hat{Q}$ 来计算目标Q值。这减少
    了目标值与预测值之间的相关性,有效抑制了训练过程中的震荡。
\end{itemize}

% 请将此处的文件名替换为您 tex 文件中 Figure 5 对应的实际文件名
\begin{figure}[htbp]
    \centering
    \includegraphics[width=1\textwidth]{Fig2/figure1.png}
    \caption{深度Q网络(DQN)架构示意图。网络接收网格状态输入,通过深度神经网络拟合动作价值函数,输出各潜在合并动作的Q值。}
    \label{fig:network_arch}
\end{figure}

\subsection{空间邻接感知合并策略}
\label{subsec:merging_strategy}

空间邻接感知合并策略是本框架的核心创新,旨在实现智能模块整合的同时,严格保留几何规则性和制造约束。该过程包含三个关键阶段:

\subsubsection{多模块组检测}
首先,通过空间邻近性分析识别潜在的合并候选对象。我们基于 Shapely 库的 \texttt{unary\_union} 操作,将每个模块的包围盒转换为多边形几何对象,并计算缓冲区(通常为 1 个单位)以处理相邻关系。该方法能够有效处理直接相邻和近邻情况,确保不遗漏任何潜在的合并机会,其时间复杂度为 $\mathcal{O}(n \log n)$。

\subsubsection{可行性检查}
检测到的每个候选组都必须通过严格的多层可行性验证:
\begin{enumerate}
    \item \textbf{尺寸约束:} 合并后的包围盒必须满足 $W \le W_{max}$ 和 $H \le H_{max}$。
    \item \textbf{完美矩形验证:} 确保合并后的几何形状内部无孔洞、无凹陷,且所有边缘正交。
    \item \textbf{面积守恒:} 验证子模块面积之和等于父模块面积(容差 0.01),以防止浮点数精度误差。
\end{enumerate}

\subsubsection{修复机制}
当合并操作未通过可行性检查时,框架将激活修复机制:
\begin{itemize}
    \item \textbf{渐进式分割:} 递归地将候选组分割为更小的子组,直到找到合法的子合并。
    \item \textbf{块重新分配:} 基于欧几里得距离和连通性分析,将无效块重新分配给最近的合法模块。
    \item \textbf{新模块生成:} 当无法重新分配时,基于剩余几何形状动态生成符合尺寸要求的新模块。
\end{itemize}


\section{实验结果与分析}
\label{sec:experiments}

本节将详细阐述所提出的空间邻接感知模块化生成框架的实验评估。首先介绍用于训练和测试的数据集生成过程及参数设置,随后定义用于量化布局质量的评价指标,最后展示模型在不同复杂度建筑轮廓下的生成结果。

\subsection{数据集生成与实验设置}
\label{subsec:dataset_setup}

由于缺乏公开的模块化建筑平面布局数据集,本研究构建了一个包含多样化建筑轮廓的合成数据集。为了模拟现实世界中复杂多变的用地条件,我们采用随机生成算法创建了训练样本。

\subsubsection{数据生成}
为了系统地验证所提出强化学习框架的有效性与鲁棒性,本研究建立了一个包含不同几何复杂度的综合数据集。该数据集具体由以下两组场景构成:

\begin{enumerate}
    \item \textbf{标准模块化场景 (Standard Modular Scenarios):} 包含源自真实住宅项目的典型建筑平面(案例 JF, LM, ZH),主要用于验证算法在常规布局标准化方面的能力。
    \item \textbf{大规模复杂场景 (Large-Scale Complex Scenarios):} 包含具有大尺度边界和高维搜索空间的复杂平面(案例 GF, XZ, HY),旨在测试算法在处理复杂空间约束时的可扩展性与稳定性。
\end{enumerate}

所有实验均在配置 Intel Core i5-13400F CPU 和 NVIDIA RTX 4070 GPU 的工作站上进行。算法
基于 Python 和 PyTorch 库实现。

\subsubsection{参数设置}
实验基于深度强化学习框架进行。在DQN训练过程中,超参数设置如下:经验回放池(Replay Buffer)大
小设定为 10,000,折扣因子 $\gamma$ 设为 0.99,学习率设为 0.001。为了平衡探索与利用,采
用 $\epsilon$-贪婪策略,$\epsilon$ 值随训练步数从 1.0 线性衰减至 0.1。此外,设置了严厉的无效
惩罚 ($R_{validity}=-50.0$) 以确保几何可行性。



\subsection{评价指标}
\label{subsec:metrics}

为了全面评估生成布局的质量,本研究主要关注以下三个核心指标,这也与奖励函数的设计保持一致:

\begin{enumerate}
    \item \textbf{覆盖率 (Coverage Rate, CR)}:衡量生成的模块化布局对原始建筑轮廓的填充程度。高覆盖率意味着更高的空间利用率和得房率。
    \item \textbf{模块标准化率 (Standardization Ratio, SR)}:定义为标准尺寸模块面积占总建筑面积的比例。该指标直接反映了布局的可制造性与经济性。
    \item \textbf{邻接完整性 (Adjacency Integrity)}:评估生成的布局是否保持了原始空间的连通性,即是否避免了将单一功能空间不合理地分割为互不连通的孤岛。
\end{enumerate}

\subsection{结果分析}
\label{subsec:results}

\begin{figure}[H]
\centering
\includegraphics[width=\textwidth]{Fig2/figure4.png} 
\caption{案例 JF 的建筑平面图及提取的模块化边界。}
\label{fig:standard_case_jf}
\end{figure}

\begin{figure}[H]
\centering
\includegraphics[width=\textwidth]{Fig2/figure5.png} 
\caption{案例 LM 的建筑平面图及提取的模块化边界。}
\label{fig:standard_case_lm}
\end{figure}

\begin{figure}[H]
\centering
\includegraphics[width=\textwidth]{Fig2/figure6.png} 
\caption{案例 ZH 的建筑平面图及提取的模块化边界。}
\label{fig:standard_case_zh}
\end{figure}

\subsubsection{标准模块化场景性能评估}
在评估的第一阶段,我们将框架应用于三个源自真实住宅项目的标准建筑
平面(案例 JF、LM 和 ZH)。图 \ref{fig:standard_case_jf}、
图 \ref{fig:standard_case_lm} 和 图 \ref{fig:standard_case_zh} 分别展示了这些
案例的建筑语境。图中显示的原始建筑平面以及提取出的正交边界轮廓,构成了模块化过
程的几何输入。需要说明的是,图中未着色的区域代表不可模块化的功能区(如楼梯间和管井),不包
含在优化范围内。

基于提取的边界约束,RL 智能体执行了布局优化。
表 \ref{tab:std_performance} 展示了在三个标准案例(JF, LM, ZH)中,本方法与人工划分基准的详细对比。
实验结果表明,智能体成功习得了符合 DfMA 原则的策略。特别是在案例 ZH 中,为了最大化标准化程度,智能体将模块类型数量 ($C_t$) 从人工方案的 19 种显著降低至 15 种(降幅 21.05\%),同时仅略微增加了 1 个模块数量。这证明了算法能够在保证覆盖率的前提下,有效权衡制造成本(模具数量)与吊装成本(模块数量)。

% \begin{table}[htbp]
%     \centering
%     \caption{标准模块化场景下的性能对比(人工划分 vs 本文方法)}
%     \label{tab:std_performance}
%     \begin{tabular}{lcccl}
%         \toprule
%         \textbf{案例} & \textbf{指标} & \textbf{人工划分} & \textbf{本文方法 (RL)} & \textbf{变化幅度} \\
%         \midrule
%         \multirow{2}{*}{Case JF} & $C_m$ (数量) & 10 & \textbf{11} & $\uparrow$ 10\% \\
%                                  & $C_t$ (类型) & 8 & \textbf{7} & $\downarrow$ 12.5\% (优化) \\
%         \midrule
%         \multirow{2}{*}{Case LM} & $C_m$ (数量) & 7 & \textbf{7} & 持平 \\
%                                  & $C_t$ (类型) & 6 & \textbf{5} & $\downarrow$ 16.67\% (优化) \\
%         \midrule
%         \multirow{2}{*}{Case ZH} & $C_m$ (数量) & 19 & \textbf{20} & $\uparrow$ 5.26\% \\
%                                  & $C_t$ (类型) & 19 & \textbf{15} & $\downarrow$ 21.05\% (显著优化) \\
%         \bottomrule
%     \end{tabular}
% \end{table}

\begin{table}[htbp]
    \centering
    \caption{标准模块化场景下的性能对比(人工划分 vs 本文 RL 方法)}
    \label{tab:std_performance}
    
    % 定义列格式:居中、定宽、垂直居中
    \begin{tabular}{ c m{0.32\linewidth}<{\centering} m{0.32\linewidth}<{\centering} m{0.22\linewidth} }
        \toprule
        \textbf{案例} & \textbf{人工划分} & \textbf{本文方法 (RL)} & \textbf{指标对比} \\
        \midrule
        \addlinespace[5pt] 
        
        % --- Case 1: JF ---
        JF 
        & 
        % 请确保路径正确,例如 Fig2/figure7_1.png
        \raisebox{-0.5\height}{\includegraphics[width=0.95\linewidth, height=2.5cm, keepaspectratio]{Fig2/figure7_1.png}}%
        & 
        \raisebox{-0.5\height}{\includegraphics[width=0.95\linewidth, height=2.5cm, keepaspectratio]{Fig2/figure7_2.png}}%
        & 
        \begin{tabular}{@{}l@{}}
            $C_m$: $10 \to \mathbf{11}$ \\ 
            \small{\textcolor{gray}{($\uparrow$ 10\%)}} \\[0.5em]
            $C_t$: $8 \to \mathbf{7}$ \\
            \small{\textcolor{gray}{($\downarrow$ 12.5\%)}} \\[0.5em]
        \end{tabular}
        \\
        \addlinespace[5pt] 
        \cmidrule(l){2-4}
        \addlinespace[5pt] 
        
        % --- Case 2: LM ---
        LM  
        & 
        \raisebox{-0.5\height}{\includegraphics[width=0.95\linewidth, height=2.5cm, keepaspectratio]{Fig2/figure7_3.png}}%
        & 
        \raisebox{-0.5\height}{\includegraphics[width=0.95\linewidth, height=2.5cm, keepaspectratio]{Fig2/figure7_4.png}}%
        & 
        \begin{tabular}{@{}l@{}}
            $C_m$: $7 \to \mathbf{7}$ \\ 
            \small{\textcolor[gray]{0.5}{(持平)}} \\[0.5em]
            
            $C_t$: $6 \to \mathbf{5}$ \\
            \small{\textcolor[gray]{0.5}{($\downarrow$ 16.67\%)}}
        \end{tabular}
        \\
        \addlinespace[5pt]
        \cmidrule(l){2-4}
        \addlinespace[5pt]
        
        % --- Case 3: ZH ---
        ZH  
        & 
        \raisebox{-0.5\height}{\includegraphics[width=0.95\linewidth, height=2.5cm, keepaspectratio]{Fig2/figure7_5.png}}%
        & 
        \raisebox{-0.5\height}{\includegraphics[width=0.95\linewidth, height=2.5cm, keepaspectratio]{Fig2/figure7_6.png}}%
        & 
        \begin{tabular}{@{}l@{}}
            $C_m$: $19 \to \mathbf{20}$ \\ 
            \small{\textcolor{gray}{($\uparrow$ 5.26\%)}} \\[0.5em]
            $C_t$: $19 \to \mathbf{15}$ \\
            \small{\textcolor{gray}{($\downarrow$ 21.05\%)}} \\[0.5em] 
        \end{tabular}
        \\
        \addlinespace[5pt]
        
        \bottomrule
    \end{tabular}
\end{table}

\subsubsection{大规模复杂场景的可扩展性}

图 \ref{fig:large_scale_boundaries} 展示了这些场景的几何复杂度和
整体尺寸。图 \ref{fig:large_scale_final} 可视化了优化后的结果。尽管边界条件极
其复杂,智能体仍成功为所有三个案例生成了有效的布局配置。
% 对应英文版 Fig 8
\begin{figure}[H]
\centering
\includegraphics[width=\textwidth]{Fig2/figure8.png} 
\caption{三个大规模复杂场景(GF, XZ, HY)的边界轮廓与尺寸定义。}
\label{fig:large_scale_boundaries}
\end{figure}

% 对应英文版 Fig 9 (中文版原文件缺失)
\begin{figure}[H]
\centering
\includegraphics[width=\textwidth]{Fig2/figure9.png} 
\caption{大规模场景下的模块化布局优化结果。不同颜色代表不同类型的模块。}
\label{fig:large_scale_final}
\end{figure}

对于具有高度非凸边界的大规模场景,表 \ref{tab:large_scale_results} 统计了生成布局的模块指标。以最复杂的 Case XZ 为例,在初始包含超过 200 个原子网格的巨大搜索空间中,算法成功将最终模块数量控制在 151 个,并保持模块类型仅为 19 种。在 Case HY 中,算法实现了极高的模块复用率,仅用 6 种模块类型即覆盖了 68 个模块区域(平均每种模具复用约 11.3 次)。

\begin{table}[htbp]
\centering
\caption{大规模复杂场景下的优化结果统计}
\label{tab:large_scale_results}
\begin{tabular}{lccc}
\toprule
\textbf{评价指标} & \textbf{案例 GF} & \textbf{案例 XZ} & \textbf{案例 HY} \\
\midrule
模块总数 ($C_m$) & 46 & 151 & 68 \\
模块类型数 ($C_t$) & 19 & 19 & 6 \\
\bottomrule
\end{tabular}
\end{table}

\subsubsection{算法特性与收敛性分析}
为了进一步探究智能体的学习动态,验证结果并非源于随机搜索,我们分析了大规模案例 XZ 优化过程中的训练日志。图 \ref{fig:convergence_curves} 展示了 100 个训练回合中,总奖励、模块数量 ($C_m$) 和模块类型多样性 ($C_t$) 的收敛曲线。

% 在这里插入刚才剪切的 figure10 代码
\begin{figure}[H]
\centering
\includegraphics[width=\textwidth]{Fig2/figure10.png} 
\caption{训练特征曲线。双轴图展示了模块数量 ($C_m$)、模块类型 ($C_t$) 以及累积奖励的优化轨迹。}
\label{fig:convergence_curves}
\end{figure}

收敛趋势揭示了三个明显的学习阶段:
\begin{enumerate}
    \item \textbf{探索阶段 (Episodes 0-30):} 由于智能体积极探索动作空间(高 $\epsilon$ 值),奖励波动较大。
    \item \textbf{策略形成阶段 (Episodes 30-70):} 观察到模块数量和类型多样性迅速下降。这表明智能体正在学习有效的空间模式,例如识别出将特定相邻块合并为更大矩形可以持续获得更高奖励。
    \item \textbf{稳定阶段 (Episodes 70-100):} 指标趋于稳定,表明策略已收敛至近似最优解。
\end{enumerate}
这一轨迹证实了所提出的空间邻接感知机制有效地修剪了无效动作空间,使智能体能够将计算资源集中在物理上可行的高回报动作上。

\section{讨论}
\label{sec:discussion}

实验结果证明了所提出的框架在解决模块化布局优化这一组合难题上的有效性。除了数值上的提升,本节将进一步分析智能体的决策逻辑,特别是奖励权重的调整如何影响面向制造与装配的设计(DfMA)中的关键权衡。

\subsection{DfMA权衡的解读与偏好学习}
\label{subsec:dfma_tradeoff}

通过对比分析标准案例的优化结果,我们发现智能体对奖励函数中的多目标配置表现出高度的敏感性。
在案例 JF 中,智能体生成了一个模块数量略有增加($C_m: 10 \to 11$)但模块类型显著减少($C_t: 8 \to 7$)的方案。这一结果并非偶然,而是奖励重塑策略的直接体现。

通过为 $R_{type}$(模块类型标准化)分配较大的权重,我们显式地激励智能体优先考虑“标准化”而非单纯的“数量最小化”。与通常依赖几何直觉来最小化切割线(从而减少 $C_m$)的人工划分不同,强化学习智能体成功习得了一种非直观的全局策略:它牺牲了局部的合并效率,以换取整体模块复用率(Module Reuse Rate)的最大化。

在预制建筑的语境下,这种策略在经济上是更优的。因为与模具制造和生产线重组相关的“多样性成本”(Cost of Variety),往往远高于运输少数几个额外标准化单元的边际物流成本。案例 JF 的行为证实,本框架不仅仅是在进行几何填充,而是能够智能地将策略与奖励函数中编码的高层级 DfMA 偏好保持一致。

\subsection{可扩展性与搜索空间管理}
\label{subsec:scalability}

在大规模复杂场景(如案例 XZ)上的可扩展性分析凸显了\textit{空间邻接感知}机制相对于传统启发式方法的优越性。
在案例 XZ 中,初始网格分解产生了超过 200 个原子块,导致有效合并对的组合搜索空间呈阶乘级增长。传统的进化算法(如遗传算法)在如此高维的空间中,由于变异算子的随机性,往往面临过早收敛或生成大量无效几何形状的问题。

相比之下,我们的 DQN 智能体利用空间邻接掩码(Spatial Adjacency Mask),有效地修剪了无效的动作空间(通过 $R_{validity}$ 惩罚),迫使探索过程仅关注几何上可行的区域。收敛轨迹表明,智能体迅速学会了拒绝随机合并,并采用了一种结构化的策略,优先形成大型、规则的矩形,即使在高度非凸的边界条件下也能实现较高的模块复用率。

\subsection{局限性与未来工作}
\label{subsec:limitations}

虽然本框架为智能模块化设计建立了稳固的基准,但仍有几个局限性值得进一步研究:
\begin{enumerate}
    \item \textbf{几何离散化约束:} 当前方法依赖于正交网格分解(OGD),这限制了其仅适用于直角多边形轮廓。处理曲线或非正交的建筑边界将需要在未来的迭代中引入基于网格(Mesh)或连续状态的表示方法。
    \item \textbf{结构物理集成:} 目前的可行性检查主要基于几何规则。未来的工作应将简化的有限元分析(FEA)集成到奖励反馈循环中,以便在合并过程中显式地优化承载能力和质心对齐。
    \item \textbf{交互式人机协作:} 尽管加权奖励函数允许进行偏好调整,但在实际应用中需要一个更具交互性的界面,允许建筑师在推理阶段动态施加“硬约束”(例如固定的核心筒位置)。
\end{enumerate}

\section{本章小结}
\label{sec:chapter_summary}

本章提出了一种针对模块化建筑平面布局设计的智能优化框架,创新性地将正交网格分解(OGD)与空间邻接感知的深度Q网络(Deep Q-Network)相结合。通过将模块划分与合并过程公式化为一个序列决策问题,我们解决了在高维设计空间中平衡空间利用率、几何规则性和制造标准化这一难题。

综合实验验证得出了以下关键结论:
\begin{enumerate}
    \item \textbf{优化有效性:} 该框架在性能上始终匹配或优于专家级的人工划分。在复杂场景(如案例 ZH)中,它通过将模块类型减少约 21\% 同时保持模块数量稳定,实现了显著的优化,成功突破了限制人类设计师的局部最优解。
    \item \textbf{标准化能力:} 通过多目标奖励机制,智能体展现了强大的面向 DfMA 的优化能力。它能够自主发现最大化模块复用率的布局策略,这对模块化建筑的经济可行性至关重要。
    \item \textbf{复杂拓扑下的鲁棒性:} 空间邻接感知策略确保了即使在具有高度不规则边界的大规模布局(如案例 XZ)中,也能实现 100\% 的几何有效性,克服了传统随机优化方法中常见的无效解问题。
\end{enumerate}

综上所述,本研究表明,当深度强化学习被赋予领域特定的空间感知能力时,能够为计算机辅助建筑设计提供一种强大且可扩展的新范式。未来的研究将集中于将此框架扩展到三维体积组装,并集成多物理场模拟,以进一步弥合概念设计与工程实现之间的鸿沟。




