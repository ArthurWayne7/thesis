\chapter{面向模块化建筑的领域特定草图生成方法研究——基于一致性约束GAN}


\section{引言}

\subsection{研究背景:模块化建筑的发展与挑战}
近年来,模块化建筑(Modular Construction)已成为建筑、工程和
施工(AEC)行业最具变革性的创新之一 \citep{lawson2012application,
Lawson2014}。作为一种颠覆性的解决方案,它旨在解决传统现场密集
型建造方法中长期存在的低效问题,有助于重振生产力、减少劳动力依赖
并最大限度地减少材料浪费 \cite{zhai2019internet}。与依赖顺序性现
场作业的传统方法不同,模块化建筑由在受控环境下预制的标准化体积单元组
装而成。这种模式实现了并行工作流,能够实施更严格的质量控制,从而显著
缩短项目交付周期并降低环境足迹。

在结构和美学层面,模块化建筑同样具有显著优势。标准化不仅简化了制造和
物流流程,还增强了结构的鲁棒性和性能可靠性。模块组装的重复逻辑和可
扩展性为空间适应性和形式表现力提供了支持。如
图 \ref{fig:modular_examples} 所示,从高层住宅到特色公建,模块的
韵律性重复不再被视为一种设计约束,而被日益公认为建筑可读性、构图秩
序乃至后工业设计表现力的来源。

\begin{figure}[htbp]
    \centering
    \includegraphics[width=1\textwidth]{Fig1/figure1.png}
    \caption{模块化建筑典型实例}
    \label{fig:modular_examples}
\end{figure}

然而,尽管模块化建筑的接受度日益提高,其设计过程仍受到场外制造和
现场安装物流所强加的刚性几何、结构及组装参数的制约。特别是立面设计,
必须在实现视觉连贯性的同时,严格遵守对齐网格、接口协调和拼板逻辑等施
工原则 \citep{zheng2024integrated}。这一现状促使研究人员探索模块
化系统的设计句法,即支配组件如何通过空间规则、组合
逻辑和类型模式进行结合的语法结构。

\subsection{现有生成式设计方法的局限性}
与此同时,人工智能(AI)技术的进步,特别是生成式深度学习的发展,为加
速早期建筑探索开辟了新途径。其中,生成对抗网络在将抽象的设
计概念转化为空间和风格连贯的视觉表征方面展现出巨大潜
力 \citep{liao2021automated}。

然而,针对模块化设计探索,现有的通用生成模型存在一个关键局限:
它们本质上对模块化建筑的句法规则(如韵律性单元重复和尺寸标准化)缺乏
感知。因此,虽然其输出结果在视觉上可能具有吸引力,但往往缺乏模块化系
统特有的底层结构逻辑,难以作为模块化建筑设计的有效灵感种子。此外,
大型生成式模型的出现虽然功能强大,但通常作为“黑箱”运行,对领域特定
约束的细粒度控制能力有限,且需要巨大的计算资源,这构成了快速、迭代
式设计构思的障碍。

\subsection{本章研究目标}
为了填补早期设计灵感阶段的这一空白,本章提出了一种轻量级且领域特定的
模块化生成对抗网络(Modular-GAN)框架。与追求通用的强大生成器不
同,本章旨在构建一个针对性工具,将模块化设计原则嵌入其中,以高效
地生成结构连贯且风格多样的草图。本章将详细阐述该框架的构建原理、核
心组件设计及其在捕捉模块化几何特征方面的有效性。

\section{模块化生成对抗网络(Modular GAN)模型构建}
\label{sec:model_construction}

本节详细论述所提生成框架的理论基础与具体实现。为了解决模块化建筑生
成中结构逻辑与形式多样性的矛盾,本研究在Wasserstein GAN的基础
上,引入了结构化的归纳偏置,并设计了专用的生成器与判别器架构。

\subsection{生成模型选择与理论基础}

\subsubsection{Wasserstein距离与建筑流形学习}
模块化建筑的高效性常被误认为是以牺牲美学多样性为代价的。本研究采
用Wasserstein生成对抗网络(WGAN)作为基础框架,原因在于其训练
的稳定性以及学习平滑、连贯数据分布的能力。

如图 \ref{fig:gan_principle} 所示,该框架遵循生成对抗网络的基
本博弈逻辑:生成器(G)接收随机噪声并尝试构建符合模块化特征的伪造
图像,而判别器(D)则通过对抗训练不断提升其区分生成图像与真实设计
方案的能力。

\begin{figure}[htbp]
    \centering
    \includegraphics[width=0.8\textwidth]{Fig1/figure3.png}
    \caption{Modular GAN 框架的基本工作原理示意图}
    \label{fig:gan_principle}
\end{figure}

WGAN在建筑生成方面的优势源于其在建筑设计流形上学习平滑
映射的能力。与传统GAN使用的Jensen-Shannon散度不同,Wasserstein距
离(又称推土机距离)为衡量建筑分布之间的差异提供了一个更有意义的
度量标准。对于建筑草图而言,即便模块布局发生微小扰动,也可能导致结
构解读上的巨大差异,而Wasserstein距离能够有效捕捉这种结构相似性。
此外,本研究采用了带有梯度惩罚的WGAN-GP架构 \cite{Gulrajani2017},通
过实施Lipschitz约束确保判别器函数的连续性,从而实现了不同模块化配
置之间更连贯的插值。

\subsubsection{深度网络中的建筑归纳偏置}
传统的卷积神经网络(CNN)具有平移等变性,这一特性虽然与模块化建筑的
重复性质相吻合,但标准卷积操作缺乏显式建模长距离结构依赖关系的能力。
为了解决这一局限性,本研究提出通过引入结构先验编码模块(SPEM),将结
构化的归纳偏置(Inductive Biases)注入生成网络中。SPEM可以被视为一种
形式的建筑注意力机制,其设计动机源于人类建筑师在设计模块化系统时,会
自然地关注结构关系和对齐逻辑。

\subsection{整体网络架构设计}

本研究设计的Modular GAN包含两个核心组件:嵌入了SPEM的生成器和用于
评估结构连贯性的判别器。网络的整体架构
如图 \ref{fig:framework_overview} 所示。

\begin{figure}[htbp]
    \centering
    \includegraphics[width=1\textwidth]{Fig1/figure2.png}
    \caption{Modular GAN 框架整体架构图:包含嵌入SPEM的生成器与多尺度判别器}
    \label{fig:framework_overview}
\end{figure}

生成器旨在将128维的潜在向量映射为 $128 \times 128$ 的高分辨率模块化
建筑草图。生成过程始于一个线性变换层,将潜在向量映
射为 $128 \times 8 \times 8$ 的特征张量。随后,网络通过四个阶
段的渐进式上采样逐步提升分辨率。为了确保生成的图像具备模块化特征,SPEM
被策略性地嵌入在每一个上采样层之后。

判别器则采用对称的下采样策略,通过一系列卷积层将空间维度逐层压
缩至 $8 \times 8$,同时通道深度扩展至512,以捕捉多尺度的判别特征。
具体的网络层级配置详见表 \ref{tab:network_arch}。

\begin{table}[htbp]
  \centering
  \caption{Modular GAN 生成器与判别器网络架构参数}
  \label{tab:network_arch}
  \small
  \begin{tabular}{@{}p{0.15\textwidth}p{0.35\textwidth}p{0.15\textwidth}p{0.35\textwidth}@{}}
    \toprule
    \textbf{模型} & \textbf{网络层配置} & \textbf{模型} & \textbf{网络层配置} \\
    \midrule
    生成器 & 
    \parbox[t]{0.35\textwidth}{\raggedright 
    nn.Linear(128, $8 \times 128 \times 8 \times 8$) \\
    Reshape to (-1, $8 \times 128$, 8, 8) \\
    nn.BatchNorm2d($8 \times 128$) \\
    nn.ReLU() \\
    UpSampleConv($8 \times 128$, $8 \times 128$, 3) \\
    SPEM($8 \times 128$) \\
    ... (后续层级逐步上采样) ... \\
    nn.Tanh() } &
    
    判别器 & 
    \parbox[t]{0.35\textwidth}{\raggedright
    nn.Conv2d(3, 128, 3, padding=1) \\
    nn.LayerNorm([128, 128, 128]) \\
    nn.ReLU() \\
    ConvMeanPool(128, $2 \times 128$, 3) \\
    ... (后续层级逐步下采样) ... \\
    Flatten \\
    nn.Linear($8 \times 128 \times 8 \times 8$, 1)} \\
    \bottomrule
  \end{tabular}
\end{table}

% \begin{table}[htbp]
%   \centering
%   \captionof{table}{Network architectures of the Generator and Discriminator in the Modular GAN Framework}
%   \label{tab:network_arch}
%   \small
%   \begin{tabular}{@{}p{0.15\textwidth}p{0.35\textwidth}p{0.15\textwidth}p{0.35\textwidth}@{}}
%     \hline
%     \textbf{Model} & \textbf{Layers} & \textbf{Model} & \textbf{Layers} \\
%     \hline
%     Generator & 
%     \parbox[t]{0.35\textwidth}{\raggedright 
%     nn.Linear(128, $8 \times 128 \times 8 \times 8$) \\
%     Reshape to (-1, $8 \times 128$, 8, 8) \\
    
%     nn.BatchNorm2d($8 \times 128$) \\
%     nn.ReLU() \\
%     UpSampleConv($8 \times 128$, $8 \times 128$, 3) \\
%     nn.BatchNorm2d($8 \times 128$) \\
%     nn.ReLU() \\
%     nn.Conv2d($8 \times 128$, $8 \times 128$, 3, padding=1) \\
%     SPEM($8 \times 128$) \\
    
%     nn.BatchNorm2d($8 \times 128$) \\
%     nn.ReLU() \\
%     UpSampleConv($8 \times 128$, $4 \times 128$, 3) \\
%     nn.BatchNorm2d($4 \times 128$) \\
%     nn.ReLU() \\
%     nn.Conv2d($4 \times 128$, $4 \times 128$, 3, padding=1) \\
%     SPEM($4 \times 128$) \\
    
%     nn.BatchNorm2d($4 \times 128$) \\
%     nn.ReLU() \\
%     UpSampleConv($4 \times 128$, $2 \times 128$, 3) \\
%     nn.BatchNorm2d($2 \times 128$) \\
%     nn.ReLU() \\
%     nn.Conv2d($2 \times 128$, $2 \times 128$, 3, padding=1) \\
%     SPEM($2 \times 128$) \\
    
%     nn.BatchNorm2d($2 \times 128$) \\
%     nn.ReLU() \\
%     UpSampleConv($2 \times 128$, 128, 3) \\
%     nn.BatchNorm2d(128) \\
%     nn.ReLU() \\
%     nn.Conv2d(128, 128, 3, padding=1) \\
%     SPEM(128) \\
    
%     nn.BatchNorm2d(128) \\
%     nn.ReLU() \\
%     nn.Conv2d(128, 3, 3, padding=1) \\
%     nn.Tanh() \\
%     Reshape to (-1, $3 \times 128 \times 128$)} &
    
%     Discriminator & 
%     \parbox[t]{0.35\textwidth}{\raggedright
%     nn.Conv2d(3, 128, 3, padding=1) \\
    
%     nn.LayerNorm([128, 128, 128]) \\
%     nn.ReLU() \\
%     nn.Conv2d(128, 128, 3, padding=1) \\
%     nn.LayerNorm([128, 128, 128]) \\
%     nn.ReLU() \\
%     ConvMeanPool(128, $2 \times 128$, 3) \\
    
%     nn.LayerNorm([$2 \times 128$, 64, 64]) \\
%     nn.ReLU() \\
%     nn.Conv2d($2 \times 128$, $2 \times 128$, 3, padding=1) \\
%     nn.LayerNorm([$2 \times 128$, 64, 64]) \\
%     nn.ReLU() \\
%     ConvMeanPool($2 \times 128$, $4 \times 128$, 3) \\
    
%     nn.LayerNorm([$4 \times 128$, 32, 32]) \\
%     nn.ReLU() \\
%     nn.Conv2d($4 \times 128$, $4 \times 128$, 3, padding=1) \\
%     nn.LayerNorm([$4 \times 128$, 32, 32]) \\
%     nn.ReLU() \\
%     ConvMeanPool($4 \times 128$, $8 \times 128$, 3) \\
    
%     nn.LayerNorm([$8 \times 128$, 16, 16]) \\
%     nn.ReLU() \\
%     nn.Conv2d($8 \times 128$, $8 \times 128$, 3, padding=1) \\
%     nn.LayerNorm([$8 \times 128$, 16, 16]) \\
%     nn.ReLU() \\
%     ConvMeanPool($8 \times 128$, $8 \times 128$, 3) \\
    
%     nn.LayerNorm([$8 \times 128$, 8, 8]) \\
%     nn.ReLU() \\
%     nn.Conv2d($8 \times 128$, $8 \times 128$, 3, padding=1) \\
%     nn.LayerNorm([$8 \times 128$, 8, 8]) \\
%     nn.ReLU() \\
%     ConvMeanPool($8 \times 128$, $8 \times 128$, 3) \\
    
%     Flatten \\
%     nn.Linear($8 \times 128 \times 8 \times 8$, 1)} \\
%     \hline
%   \end{tabular}
% \end{table}

\subsection{生成器基础组件详解}

为了支撑高效的图像生成,生成器的分辨率提升管道构建在两个基础组
件之上:上采样卷积单元(UpSampleConv)和残差块(Residual Block)。

\textbf{1. 上采样卷积单元(UpSampleConv):} 
如图 \ref{fig:upsample_conv} 所示,该单元通过两阶段过程将特征图
分辨率从 $C \times H \times W$ 提升至 $C \times 2H \times 2W$。
首先,利用通道复制机制(Channel Repeat)将特征维度扩展四倍,建立
冗余表征;随后,应用深度到空间的像素重组(Pixel Shuffle)操
作 \cite{Shi2016} 将通道信息重新排列至空间维度。相比传统的插值上
采样,这种方法能够通过反向传播学习自适应的上采样核,并在最后的卷积
层中平滑特征,有效减少了生成图像中的棋盘格伪影。

\begin{figure}[htbp]
    \centering
    \includegraphics[width=0.8\textwidth]{Fig1/figure4.png}
    \caption{上采样卷积单元架构:展示通道复制与像素重组变换过程}
    \label{fig:upsample_conv}
\end{figure}

\textbf{2. 残差块(Residual Block):} 
如图 \ref{fig:residual_block} 所示,生成器集成了残差块以增强
特征学习和梯度流稳定性。残差块包含主变换路径和捷径(Shortcut)路径。主
路径包含一系列归一化、激活和卷积操作,而捷径路径则执行恒等映射。两者
的逐元素相加使得网络能够同时学习增量特征变换和保留原始信息,确保了深
层架构下的梯度传播稳定性 \cite{He2016}。

\begin{figure}[htbp]
    \centering
    \includegraphics[width=0.8\textwidth]{Fig1/figure5.png}
    \caption{残差块详细架构:展示主变换路径与捷径连接}
    \label{fig:residual_block}
\end{figure}

\subsection{结构先验编码模块(SPEM)原理与实现}

针对模块化建筑图像中固有的几何重复性与结构刚性特征,本研究设计了结
构先验编码模块(SPEM)。不同于传统的特征增强策略,SPEM引入了一条
轻量级但具备结构感知能力的修正路径,能够在保持计算效率的同时选择性
地增强空间模式。

\subsubsection{SPEM 内部机制}
如图 \ref{fig:spem_arch} 所示,SPEM 的内部处理流程包含三个关键步骤:

1.  **通道降维(Channel Reduction):** 首先利用 $1 \times 1$ 卷
积对输入特征图进行降维。这一操作构建了一个信息瓶颈,迫使网络滤除噪声,
聚焦于最核心的结构化特征。
    
2.  **结构先验编码(Structure Prior Encoder, SPE):** 降维后的特
征进入SPE子网络。该子网络采用标准 $3 \times 3$ 卷积配合深度卷
积(Depthwise Convolution)。从信号处理的角度来看,SPE中的深度卷
积充当了一组空间感知滤波器,能够自适应地响应不同的建筑模式。这
对于模块化建筑尤为重要,因为结构模式往往在保持底层几何关系的同时
,在尺度和方向上存在变化。

3.  **特征恢复与残差融合:** 经过结构编码后,利用第二
个 $1 \times 1$ 卷积将特征图恢复至原始维度。为了保持原始语义
信息的连续性,SPEM采用了残差连接机制,将增强后的特征与原始输入相加。

SPEM在第 $l$ 层的前向传播计算过程可由公式(\ref{eq:reduced})
至(\ref{eq:out})表示:

\begin{align}
  F_{\text{reduced}} &= \text{Conv}_{1 \times 1}(F_l) \label{eq:reduced} \\
  F_{\text{encoded}} &= \text{SPE}(F_{\text{reduced}}) \label{eq:encoded} \\
  F_{\text{restored}} &= \text{Conv}_{1 \times 1}(F_{\text{encoded}}) \label{eq:restored} \\
  F_{\text{out}} &= F_l + F_{\text{restored}} \label{eq:out}
\end{align}

其中,$F_l$ 表示第 $l$ 层的输入特征图,$\text{SPE}(\cdot)$ 表示
结构先验编码器操作。

\begin{figure}[htbp]
    \centering
    \includegraphics[width=0.8\textwidth]{Fig1/figure6.png}
    \caption{结构先验编码器 (SPE) 网络架构图}
    \label{fig:spem_arch}
\end{figure}

综上所述,通过在生成过程中引入SPEM,实际上是向网络注入了一
种对几何敏感的注意力机制,引导网络在不同尺度上保持模块的规律
性。这种设计使得生成器产出的建筑草图在重复单元的对齐精度、模块
边界的锐利度以及整体布局的空间连贯性上均有显著提升。


\section{融合模块化一致性约束的风格迁移机制}
\label{sec:style_transfer}

在模块化建筑的生成设计中,美学风格的融合绝非仅限于视觉纹理与线条特征
的迁移,更为关键的是必须保持空间模块的几何一致性与重复性逻
辑 \cite{Brock2019}。为了实现这一目标,本节提出了一种增强型的建筑
风格融合模块(Architectural Fusion Module, AFM)。该模块基
于STROTSS(Style Transfer by Relaxed Optimal Transport and 
Self-Similarity)框架 \cite{Li2021} 构建,并创新性地引入了模块化
一致性损失(Modular Consistency Loss, MCL),以确保在风格化过程中
维持建筑固有的模块化结构 \cite{Chen2021}。

\subsection{风格融合模块工作流程}

本研究提出的风格融合框架采用多输入、多阶段的优化流程,如
图 \ref{fig:fusion_workflow} 所示。该模块接收两个主要输入:
\begin{enumerate}
    \item \textbf{内容图像(Content Image):} 由前述Modular GAN
	生成器输出的模块化建筑草图,它提供了建筑的基本体量、模块划分和透
	视关系。
    \item \textbf{风格参考图(Style Reference):} 用户提供的包
	含特定美学特征(如包豪斯风格、传统纹理等)的图像。
\end{enumerate}

\begin{figure}[htbp]
    \centering
    \includegraphics[width=1\textwidth]{Fig1/figure7.png}
    \caption{模块化建筑风格融合框架工作流程图}
    \label{fig:fusion_workflow}
\end{figure}

融合网络旨在合成一个新的建筑草图,使其在保留前者空间布局的同时,
习得后者的视觉语义风格。如图 \ref{fig:fusion_schematic} 所示,整
个优化过程由一个复合损失函数驱动,该函数由三部分组成:内容损失、风格
损失以及本研究提出的模块化一致性损失。总损失函数定义如下:

\begin{equation}
  \mathcal{L}_{\text{total}} = \mathcal{L}_{\text{style}} + w_c \mathcal{L}_{\text{content}} + w_m \mathcal{L}_{\text{mcl}} 
  \label{eq:total-loss}
\end{equation}

其中,$w_c$ 和 $w_m$ 分别为内容损失和模块化一致性损失的权重系数,用于
平衡结构保持与风格迁移的程度。

\begin{figure}[htbp]
    \centering
    \includegraphics[width=1\textwidth]{Fig1/figure8.png}
    \caption{建筑风格融合模块整体架构示意图}
    \label{fig:fusion_schematic}
\end{figure}

\subsection{基础损失函数:内容与风格的解耦}

为了确保生成的草图既能维持高层级的布局结构,又能获得参考图像的风格
语义,本研究在预训练的VGG16网络 \cite{Simonyan2015} 构建的共享特
征空间中计算损失函数。

\subsubsection{基于自相似性的内容损失}
内容损失 $\mathcal{L}_{\text{content}}$ 的设计旨在保留生成
草图 $I_g$ 的整体空间布局和设计意图。受局部自相似性保持机
制 \cite{Li2021} 的启发,我们在 $N$ 个空间位置提取超
列(Hypercolumn)描述符 \cite{Hariharan2015}。

设 $A_{g} = \{a_g^i\}_{i=1}^N$ 和 $A_{c} = \{a_c^i\}_{i=1}^N$ 分别
为生成图像和内容参考图像的特征描述符集合。对每个描述符进行归一
化后,计算成对的余弦距离以构建局部自相似矩阵 \cite{Shechtman2007}:

\begin{equation}
  D_{g}^{ij} = 1 - \frac{a_g^i \cdot a_g^j}{\|a_g^i\| \|a_g^j\|}, \quad
  D_{c}^{ij} = 1 - \frac{a_c^i \cdot a_c^j}{\|a_c^i\| \|a_c^j\|}
  \label{eq:distance}
\end{equation}

内容损失定义为这两个距离矩阵之间的Frobenius范数平方:

\begin{equation}
  \mathcal{L}_{\text{content}} = \frac{1}{N^2} \sum_{i=1}^{N} \sum_{j=1}^{N} \left( D_{g}^{ij} - D_{c}^{ij} \right)^2
  \label{eq:content-loss}
\end{equation}

该公式强制网络保留语义区域之间的局部关系结构 \cite{Gatys2016},从而确
保生成的建筑布局与原始草图在拓扑上保持一致。

\subsubsection{基于最优传输的风格损失}
风格损失 $\mathcal{L}_{\text{style}}$ 侧重于对齐生成图像 $I_g$ 与风
格参考图 $I_s$ 之间的视觉特征(如纹理、线条强度和材质感)。为此,本研
究采用松弛推土机距离(Relaxed Earth Mover's Distance, REMD)作为
特征集对齐的主要度量标准 \cite{Cuturi2013},并结合矩匹配
(Moment Matching)以保持强度一致性 \cite{Huang2017}。

设 $A_{g} = \{a_g^i\}_{i=1}^{N_g}$ 和 
$A_{s} = \{a_s^j\}_{j=1}^{N_s}$ 分别为从 $I_g$ 和 $I_s$ 中提取
的特征集。两组特征集之间的REMD定义为:

\begin{equation}
  \text{REMD}(A_g, A_s) = \max\left\{ \frac{1}{N_g} \sum_{i=1}^{N_g} \min_{j} \|a_g^i - a_s^j\|_2, \;
  \frac{1}{N_s} \sum_{j=1}^{N_s} \min_{i} \|a_s^j - a_g^i\|_2 \right\}
  \label{eq:remd}
\end{equation}

为了进一步确保全局风格的连贯性,我们在损失函数中加入了均值和协方差约
束。完整的风格损失公式如下:

\begin{equation}
  \mathcal{L}_{\text{style}} = \text{REMD}(A_g, A_s) + \lambda_\mu \|\mu_g - \mu_s\|_2^2 + \lambda_\Sigma \|\Sigma_g - \Sigma_s\|_F^2
  \label{eq:style-loss}
\end{equation}

其中,$\mu$ 和 $\Sigma$ 分别代表特征集的均值向量和协方差矩
阵,$\lambda_\mu$ 和 $\lambda_\Sigma$ 为平衡系数。这种设计确保了
在融合过程中,结构细节和全局风格特征都能被忠实地迁移 \cite{Li2017}。

\subsection{模块化一致性损失 (MCL)}

针对模块化建筑的独特属性——即标准化单元的重复性和严格的空间对齐要求—
—本研究提出了模块化一致性损失(Modular Consistency Loss, MCL)。该
损失函数通过度量生成图像中不同模块化区域深层特征表示的相似性,来强制
实现模块表达的统一性。

具体而言,生成的草图首先被划分为 $N$ 个局部模块化补
丁(Patches) $\{M_1, M_2, ..., M_N\}$。对于每一个补丁 $M_i$,利用
VGG19网络的中间层(如 \texttt{conv4\_2})提取其特征
嵌入 $F(M_i)$ \cite{Simonyan2015}。模块化一致性损失计算为所有补
丁对之间的平均余弦相异度,并引入空间邻近度进行加权:

\begin{equation}
  \mathcal{L}_{\text{MCL}} = \frac{1}{N(N-1)} \sum_{i=1}^{N} \sum_{\substack{j=1 \\ j\neq i}}^{N} w_{ij} \cdot \left(1 - \frac{F(M_i) \cdot F(M_j)}{\|F(M_i)\| \|F(M_j)\|}\right)
  \label{eq:mcl-loss-improved}
\end{equation}

其中,$w_{ij} = \exp\left(-\frac{\|p_i - p_j\|^2}{\sigma^2}\right)$ 是基于补丁中心坐标 $p_i$ 和 $p_j$ 的空间高斯核函数。该公式有效地促进了模块单元的一致性表达,防止了模块化布局中常见的结构变形或过度风格化问题。

\begin{figure}[htbp]
    \centering
    \includegraphics[width=1\textwidth]{Fig1/figure9.png}
    \caption{模块化一致性损失 (MCL) 机制架构图}
    \label{fig:mcl}
\end{figure}

如图 \ref{fig:mcl} 所示,MCL机制系统地处理模块化补丁,通过特征提
取和相似度计算,确保重复的建筑元素在融合过程中保持一致。通过用这种模
块化一致性机制替代传统的基于线条的正则化,融合模块能够更好地契合模块
化施工的需求,兼顾风格丰富性与结构连贯性。最终生成的建筑草图不仅反映
了用户的风格意图,更严格遵守了模块化建筑系统的重复与对齐逻
辑 \cite{liu2021swin}。

\subsection{MCL的优化动力学分析}

MCL的引入根本性地改变了风格迁移问题的优化景
观(Optimization Landscape)。与传统的内容和风格损失不同,后者通常独
立作用于图像的不同方面,而MCL引入了空间分布区域之间的耦合关系,构建
了一个更为复杂但在建筑学意义上更为合理的目标函数 \cite{Zhao2024}。

MCL中的空间加权机制可以被理解为一种“建筑局部性偏置”
(Architectural Locality Bias):即空间上相邻的模块被期望比远距离模
块表现出更高的一致性。这一设计选择反映了模块化建筑的物理约束,即相邻模
块必须保持精确的接口对齐以满足施工
要求 \cite{Huang20172, Barz2021}。

从优化的角度来看,MCL在损失景观中引入了额外的临界点,这些点对应于具有
建筑一致性的解。高斯加权确保了这些临界点之间的平滑过渡,从而促进了融合
过程中的稳定收敛 \cite{Chen2022b}。

\section{实验设置与数据集构建}
\label{sec:experiments_setup}

为了验证本章提出的Modular GAN框架在生成质量、结构连贯性以及风格迁移
方面的有效性,本节详细阐述了实验所用的数据集构建流程、预处理方法以及具体
的网络训练参数设置。

\subsection{模块化建筑草图数据集的构建与预处理}

数据是驱动深度生成模型学习领域特征的基础。由于目前缺乏公开的、专门针
对模块化建筑的高质量草图数据集,本研究采用了Qian等人 \cite{Qian2022} 建立
的系统化数据处理流程,并针对模块化建筑的特性进行了扩展和优化。

\subsubsection{数据采集与筛选}
研究团队首先通过网络爬虫技术从公开建筑数据库中采集了大量真实建筑图像。为
了确保训练数据的质量并使其适用于模块化生成任务,我们实施了严格的质量
控制筛选标准:
\begin{enumerate}
    \item \textbf{形态完整性:} 图像必须包含完整的建筑立面或体量,避免
	局部遮挡或过度裁剪,以保证模型能够学习到全局的结构逻辑。
    \item \textbf{模块化特征显著性:} 筛选出的建筑必须表现出明显的模
	块化特征,如清晰的单元边界、重复的结构模式以及标准化的接口细节。
    \item \textbf{视觉清晰度:} 剔除背景杂乱、光照条件极端或分辨
	率过低的图像,确保后续边缘提取的准确性。
\end{enumerate}

\subsubsection{基于XDOG的草图风格化处理}
为了模拟概念设计阶段的草图效果,并减少色彩和光照对结构学习的干扰,本研
究采用扩展高斯差分算
子(eXtended Difference-of-Gaussians, XDOG)\cite{Winnemoller2012} 将
采集的照片转化为线条清晰的建筑草图。

XDOG算法通过调整高斯模糊的参数 $\sigma$ 和阈值 $\epsilon$,能够灵活控
制线条的粗细和风格化程度。本实验沿用了Qian等人验证过的参数配置,以保留
建筑的关键几何轮廓同时去除冗余的纹理噪声。经过处理,最终构建了一个包
含1200张高质量模块化建筑草图的基础数据
集。如图 \ref{fig:dataset_samples} 所示,处理后的草图有效地保留了模
块化单元的几何特征和整体形态。

\begin{figure}[htbp]
    \centering
    \includegraphics[width=1\textwidth]{Fig1/figure10.png}
    \caption{模块化建筑数据集样本示例:原始图像(上)与经过XDOG处理后的草图(下)}
    \label{fig:dataset_samples}
\end{figure}

\subsection{实验环境与网络训练参数设置}

\subsubsection{实验硬件与软件环境}
本章涉及的所有实验均在统一的计算平台上完成,以确保结果的可复现性。硬件
配置包括 Intel Core i5-13400F CPU、32GB DDR4 内存以
及 NVIDIA GeForce RTX 4070 GPU(12GB 显存)。软件环境基于 Python 编程
语言和 PyTorch 深度学习框架实现。

\subsubsection{网络优化与超参数配置}
为了保证Modular GAN框架的稳定收敛,针对不同模块采用了差异化的优化策略:

\begin{itemize}
    \item \textbf{生成与判别模块优化:} 采用Adam优
	化器 \cite{Kingma2014} 对生成器和判别器进行参数更新。经过参数敏感性
	分析(详见第4章讨论),学习率设定为 $0.001$,动量参数设定
	为 $\beta_1=0.9$ 和 $\beta_2=0.999$。这一配置在保持训练稳定
	性的同时,能够有效避免模式崩塌。
    
    \item \textbf{风格融合模块优化:} 在风格迁移阶段,采用RMSprop优
	化器对输入图像进行迭代更新,学习率设定为 $0.0025$。为了平衡结构
	保持与风格注入,损失函数的权重系数设定为:内容损失权重 $w_c = 16$,模
	块化一致性损失权重 $w_m = 0.005$。
    
    \item \textbf{数据增强策略:} 考虑到基础数据量相对有限,为了防止
	过拟合并提升模型的泛化能力,我们在训练过程中引入了在线数据增强技
	术。具体操作包括随机水平翻转、中心裁剪以及范
	围在 $[-10^\circ, 10^\circ]$ 内的随机旋转。通过数据增强,训练集被
	扩充至4800张,显著提升了模型对不同视角和构图的适应性。
\end{itemize}

\begin{table}[htbp]
  \centering
  \caption{Modular GAN 训练超参数设置一览表}
  \label{tab:hyperparameters}
  \begin{tabular}{lcc}
    \toprule
    \textbf{参数名称} & \textbf{涉及模块} & \textbf{数值} \\
    \midrule
    基础学习率 (Learning Rate) & Generator/Discriminator & 0.001 \\
    Adam $\beta_1$ & Generator/Discriminator & 0.9 \\
    Adam $\beta_2$ & Generator/Discriminator & 0.999 \\
    Batch Size & Training & 32 \\
    Epochs & Training & 200 \\
    RMSprop Learning Rate & Fusion Module & 0.0025 \\
    内容损失权重 ($w_c$) & Fusion Module & 16 \\
    MCL损失权重 ($w_m$) & Fusion Module & 0.005 \\
    \bottomrule
  \end{tabular}
\end{table}

\section{结果分析与验证}
\label{sec:results_analysis}

为了全面评估Modular GAN框架的性能,本节从定量指标和定性视觉效果两个维度
展开分析。首先通过FID指标评估生成图像的分布质量及参数敏感性;其次,展示
模型在不同风格下的生成效果;随后,通过与真实建筑案例的对比验证生成的几何
可行性;最后,通过消融实验深入探讨模块化一致性损失(MCL)在维持结构
完整性方面的关键作用。

\subsection{生成质量的定量评估与参数敏感性分析}

\subsubsection{评价指标:Fréchet Inception Distance (FID)}
本研究采用Fréchet Inception Distance (FID) \cite{Heusel2017} 作为主要
的定量评价指标。FID通过计算真实图像分布与生成图像分布在Inception-v3网络
特征空间中的距离来衡量生成质量,其计算公式如下:

\begin{equation}
  \text{FID}(r, g) = \|\mu_r - \mu_g\|^2 + \text{Tr}(\Sigma_r + \Sigma_g - 2(\Sigma_r \Sigma_g)^{1/2})
  \label{eq:fid}
\end{equation}

其中,$\mu_r, \mu_g$ 分别表示真实图像和生成图像特征向量的均
值,$\Sigma_r, \Sigma_g$ 表示相应的协方差矩阵。FID数值越低,表明生成图像
在视觉质量和多样性上越接近真实数据分布。在本实验中,我们随机生成了3,000张
样本用于计算FID分数,以确保评估结果的统计显著性。

\subsubsection{参数敏感性分析}
优化器的超参数设置对GAN的收敛质量有显著影响。为了探究最佳训练配置,本研究
对比了Adam优化器中不同动量参数 $\beta_1$ 和 $\beta_2$ 组合下的FID得
分。如表 \ref{tab:fid_comparison} 所示,实验结果表明:

\begin{itemize}
    \item 当 $(\beta_1, \beta_2) = (0.0, 0.9)$ 时,FID高达117.21,说明过低的动量导致训练不稳定,生成分布与真实分布差异较大。
    \item 当 $(\beta_1, \beta_2) = (0.9, 0.999)$ 时,FID降至最低的87.65,相比前者提升了约31.1\%。这表明较高的动量参数有助于生成器在复杂的模块化几何空间中找到更优的鞍点。
    \item 进一步调整为 $(0.5, 0.999)$ 时,性能有所回落(FID=101.33),证实了当前参数选择的最优性。
\end{itemize}

\begin{table}[htbp]
  \centering
  \caption{不同Adam优化器参数下的Modular GAN生成质量(FID)对比}
  \label{tab:fid_comparison}
  \begin{tabular}{cc}
    \toprule
    \textbf{优化器参数 ($\beta_1, \beta_2$)} & \textbf{FID (越低越好)} \\
    \midrule
    (0.0, 0.9) & 117.21 \\
    \textbf{(0.9, 0.999)} & \textbf{87.65} \\
    (0.5, 0.999) & 101.33 \\
    \bottomrule
  \end{tabular}
\end{table}

\subsection{模块化草图生成的定性评价}

图 \ref{fig:generated_samples} 展示了Modular GAN在训练收敛后生
成的代表性模块化建筑草图。从视觉效果来看,模型成功习得了模块化建筑的核
心句法规则(Syntactic Rules):

\begin{enumerate}
    \item \textbf{单元重复性:} 生成的草图展现了标准单元的系统性重复,
	窗户、阳台等构件在立面上呈现出严格的韵律感,而非随机的纹理堆砌。
    \item \textbf{接口对齐:} 模块之间的连接边界清晰锐利,垂直方向的
	承重结构与水平方向的楼板线保持了良好的正交对齐,体现了SPEM模块在捕
	捉长距离结构依赖方面的有效性。
    \item \textbf{整体协调性:} 建筑体量比例协调,透视关系准确,具备
	作为概念设计方案的可读性。
\end{enumerate}

\begin{figure}[htbp]
    \centering
    \includegraphics[width=1\textwidth]{Fig1/figure11.png}
    \caption{Modular GAN 生成的模块化建筑草图示例:展示了清晰的模块边界与结构韵律}
    \label{fig:generated_samples}
\end{figure}

\subsection{风格融合效果分析}

为了验证架构融合模块(AFM)的泛化能力,我们将生成的模块化底图与多种
具有显著差异的建筑风格进行了融合测试。

\subsubsection{现代与未来主义风格融合}
如图 \ref{fig:fusion_honeycomb} 所示,以
“蜂巢建筑(Honeycomb Building)”为参考风格时,模型成功将六边形的几
何母题映射到了正交的模块化网格上,同时保留了底图的体量结构。类似地,
图 \ref{fig:fusion_capsule} 展示了以日本“中银胶
囊塔(Nakagin Capsule Tower)”为参考的融合结果,生成的草图精准地复现
了圆形窗洞和独立的胶囊单元质感,证明了模型对特殊模块化形式的适应能力。

\begin{figure}[htbp]
    \centering
    \includegraphics[width=1\textwidth]{Fig1/figure12.png}
    \caption{基于蜂巢建筑风格的融合生成结果}
    \label{fig:fusion_honeycomb}
\end{figure}

\begin{figure}[htbp]
    \centering
    \includegraphics[width=1\textwidth]{Fig1/figure13.png}
    \caption{基于中银胶囊塔风格的融合生成结果}
    \label{fig:fusion_capsule}
\end{figure}

\subsubsection{经典风格与纹理迁移}
图 \ref{fig:fusion_bauhaus} 展示了以包豪斯(Bauhaus)风格为参考的融合
效果。模型不仅迁移了现代主义简洁的线条和光影质感,还强化了水平向的构
图逻辑。这些结果表明,AFM模块能够在不破坏模块化结构完整性的前提下,赋予设
计方案丰富的风格语义,支持跨文化、跨时代的设计探索。

\begin{figure}[htbp]
    \centering
    \includegraphics[width=1\textwidth]{Fig1/figure14.png}
    \caption{基于包豪斯风格的融合生成结果}
    \label{fig:fusion_bauhaus}
\end{figure}

\subsection{真实案例对比与可行性验证}

为了验证生成草图的工程可行性,本研究将AI生成的方案与现实世界中的已建项目
进行了对比分析。

如图 \ref{fig:real_world_comparison1} 所示,生成的公建类草图在体块堆叠
逻辑、悬挑比例以及开窗模数上,与真实的现代模块化办公楼呈现出惊人的一致性。
图 \ref{fig:real_world_comparison2} 则展示了住宅类草图与模块化别墅的对
比,生成的方案准确捕捉了错层退台和独立单元拼接的特征。

这种高度的几何一致性验证了本研究提出的SPEM和MCL机制并非仅仅是在模仿图
像纹理,而是真正捕捉到了支配模块化建造的底层几何逻辑(Geometric Logic)。
生成的草图具备转化为BIM模型或深化设计方案的潜力,而不仅仅是艺术构想。

\begin{figure}[htbp]
    \centering
    \includegraphics[width=1\textwidth]{Fig1/figure16.png}
    \caption{生成草图与真实模块化公建案例的几何特征对比}
    \label{fig:real_world_comparison1}
\end{figure}

\begin{figure}[htbp]
    \centering
    \includegraphics[width=1\textwidth]{Fig1/figure17.png}
    \caption{生成草图与真实模块化住宅案例的结构逻辑对比}
    \label{fig:real_world_comparison2}
\end{figure}

\subsection{模块化一致性损失 (MCL) 的消融实验}

为了量化模块化一致性损失(MCL)在风格融合中的具体贡献,我们设计了消融
实验(Ablation Study)。图 \ref{fig:ablation_study} 展示了移除不同损失
函数分量后的生成结果对比:

\begin{itemize}
    \item \textbf{无内容损失 (c):} 图像结构完全崩塌,无法保持原始建筑的轮廓。
    \item \textbf{无风格损失 (d):} 生成结果退化为对底图的简单重建,未能获得参考图像的纹理特征。
    \item \textbf{无MCL损失 (e):} 这是最关键的对比组。在移除MCL后,虽然风格和大致轮廓得以保留,但模块的重复性特征变得模糊,局部线条出现扭曲,单元之间的接口对齐(Alignment)严重缺失。这证实了传统神经风格迁移方法会导致结构失真。
    \item \textbf{完整模型 (f):} 引入MCL后,生成的草图在保持风格特征的同时,显著修复了模块边界的直线度和排列的规律性,结构清晰度得到质的提升。
\end{itemize}

这一结果有力地证明了MCL机制通过施加特征空间的一致性约束,有效解
决了风格迁移中“结构漂移”的问题,对于模块化建筑设计至关重要。

\begin{figure}[H]
    \centering
    \subfloat[content image]{\label{fig:content_image}
        \includegraphics[width=0.15\textwidth, height=4cm, keepaspectratio]{Fig1/figure18.png}}
    \hfill
    \subfloat[style image]{\label{fig:style_image}
        \includegraphics[width=0.15\textwidth, height=4cm, keepaspectratio]{Fig1/figure19.png}}
    \hfill
    \subfloat[no content loss]{\label{fig:no_content_loss}
        \includegraphics[width=0.15\textwidth, height=4cm, keepaspectratio]{Fig1/figure20.png}}
    \hfill
    \subfloat[no style loss]{\label{fig:no_style_loss}
        \includegraphics[width=0.15\textwidth, height=4cm, keepaspectratio]{Fig1/figure21.png}}
    \hfill
    \subfloat[no mcl loss]{\label{fig:no_line_loss}
        \includegraphics[width=0.15\textwidth, height=4cm, keepaspectratio]{Fig1/figure22.png}}
    \hfill
    \subfloat[all loss funcs]{\label{fig:with_all_loss}
        \includegraphics[width=0.15\textwidth, height=4cm, keepaspectratio]{Fig1/figure23.png}}
    \caption{不同损失函数配置下的消融实验结果对比:(a) 内容图; (b) 风格图; (c) 无内容损失; (d) 无风格损失; (e) 无MCL损失; (f) 完整模型}
    \label{fig10}
\end{figure}

\subsection{交互式设计工具演示}

为了验证本框架在实际设计工作流中的应用价值,本研究开发了一个交
互式客户端原型。如图 \ref{fig:client_demo} 所示,设计师可以通过该界面选
择基础的模块化布局,并指定风格参考图(如传统中式塔楼或现代胶囊风格)。
系统能够实时生成多个风格各异但结构统一的设计变体。这种“人在回
路”(Human-in-the-loop)的设计模式,将AI从单纯的生成器转变为辅助
设计师进行方案发散的智能伙伴。

\begin{figure}[htbp]
    \centering
    \includegraphics[width=1\textwidth]{Fig1/figure15.png}
    \caption{交互式模块化建筑设计助手客户端界面及生成示例}
    \label{fig:client_demo}
\end{figure}

\section{本章小结}

本章针对模块化建筑设计中面临的“标准化与多样性”矛盾,提出了一种领域特
定的Modular GAN生成框架。主要工作总结如下:

\begin{enumerate}
    \item \textbf{模型构建:} 设计了嵌入结构先验编码模块(SPEM)的轻
	量级生成对抗网络,通过在特征提取阶段强化几何对齐和重复性模式,使
	得生成的草图天生具备模块化结构逻辑。
    \item \textbf{风格融合机制:} 提出了一种创新的模块化一致性损
	失(MCL),解决了传统风格迁移算法破坏建筑结构严谨性的问题,实现了在保持模块化接口对齐的前提下进行多样化的风格探索。
    \item \textbf{实证验证:} 实验结果表明,该模型在FID指标上优于
	基准参数配置(87.65 vs 117.21),且在定性评估中展现了对包豪斯、胶
	囊塔等多种复杂风格的适应性。消融实验进一步证实了MCL在维持结构完整
	性方面的核心作用。
\end{enumerate}

本章的研究解决了模块化建筑“外观生成”的问题,为早期的方案构思提供了丰
富的素材。然而,建筑设计不仅是外观的艺术,更是空间的科学。生成的草
图虽然具有视觉上的合理性,但其内部的平面功能布局是否符合逻辑、能否
满足实际的使用需求,仍需进一步探讨。因此,下一章将深入建筑内部,研
究基于深度强化学习的模块化平面布局优化方法,以实现从“外表皮”到“内空
间”的智能生成闭环。